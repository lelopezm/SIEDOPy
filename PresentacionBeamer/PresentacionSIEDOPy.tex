\documentclass[12pt,letterpaper]{beamer}
\usepackage[utf8]{inputenc}
\usepackage[spanish]{babel}
\usepackage[T1]{fontenc}
\usepackage{amsmath}
\usepackage{amssymb}
\usepackage{graphicx}
\usepackage{graphics}
\usepackage{color}
\usepackage{xcolor}
\usepackage{parskip}
\usepackage{hyperref}
\definecolor{azulp}{rgb}{0.035,0.239,0.443} %color azul de la presentación
\definecolor{doradop}{rgb}{0.839,0.7216,0.3804} %color dorado de la presentación
%información de la presentación
\title{SIEDOPy}
\author{Semillero de Investigación ALTENUA}
\date{Universidad de Caldas}
\usefonttheme{professionalfonts}
\usecolortheme[rgb={0.035,0.239,0.443}]{structure}
\usetheme{Madrid}
\begin{document}
%Portada
{
\usebackgroundtemplate{\includegraphics[width=\paperwidth,height=\paperheight]{images/portada.pdf}}
\setbeamertemplate{footline}{\usebeamertemplate*{minimal footline}}
\begin{frame}
\end{frame}
}

{
	\usebackgroundtemplate{\includegraphics[width=\paperwidth,height=\paperheight]{images/fondo.pdf}}
	\setbeamertemplate{footline}{\usebeamertemplate*{minimal footline}}
	\begin{frame}
	\end{frame}
}
%%__________________________________________________________________________________________________
%AQUI EMPIEZA LA PRESENTACIÓN
\begin{frame}
	\textcolor{azulp}{Este es un texto con el mismo color de la presentación}
	
	Este es un bloque:
	\begin{block}{Bloque}
		La ecuación
		\[
		ax^2+bx+c=0
		\]
	\end{block}
	
	\begin{block}{\textcolor{doradop}{\bf Bloque}}
		La ecuación
		\[
		ax^2+bx+c=0
		\]
	\end{block}

	Esta es una alerta:
	\begin{alertblock}{Alerta}
		La ecuación
		\[
		ax^2+bx+c=0
		\]
	\end{alertblock}
\end{frame}

\begin{frame}{Este es un título para la diapositiva}
\begin{figure}
	\centering
	\includegraphics[scale=0.4]{images/portada.pdf}
	\caption{Ejemplo de una imagen}
\end{figure}	
\end{frame}
\end{document}